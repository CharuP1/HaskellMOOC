\documentclass[11pt]{amsart}
\usepackage{geometry}                % See geometry.pdf to learn the layout options. There are lots.
\geometry{letterpaper}                   % ... or a4paper or a5paper or ... 
%\geometry{landscape}                % Activate for for rotated page geometry
%\usepackage[parfill]{parskip}    % Activate to begin paragraphs with an empty line rather than an indent
%\usepackage{hyperref}
\usepackage{graphicx}
\usepackage{amssymb}
\usepackage{epstopdf}
\DeclareGraphicsRule{.tif}{png}{.png}{`convert #1 `dirname #1`/`basename #1 .tif`.png}
% From https://gist.github.com/cormacrelf/3760427
\usepackage{color}
\usepackage{fancyvrb}
\DefineShortVerb[commandchars=\\\{\}]{\|}
\DefineVerbatimEnvironment{Highlighting}{Verbatim}{commandchars=\\\{\}}
% Add ',fontsize=\small' for more characters per line
\newenvironment{Shaded}{}{}
\newcommand{\KeywordTok}[1]{\textcolor[rgb]{0.00,0.44,0.13}{\textbf{{#1}}}}
\newcommand{\DataTypeTok}[1]{\textcolor[rgb]{0.56,0.13,0.00}{{#1}}}
\newcommand{\DecValTok}[1]{\textcolor[rgb]{0.25,0.63,0.44}{{#1}}}
\newcommand{\BaseNTok}[1]{\textcolor[rgb]{0.25,0.63,0.44}{{#1}}}
\newcommand{\FloatTok}[1]{\textcolor[rgb]{0.25,0.63,0.44}{{#1}}}
\newcommand{\CharTok}[1]{\textcolor[rgb]{0.25,0.44,0.63}{{#1}}}
\newcommand{\StringTok}[1]{\textcolor[rgb]{0.25,0.44,0.63}{{#1}}}
\newcommand{\CommentTok}[1]{\textcolor[rgb]{0.38,0.63,0.69}{\textit{{#1}}}}
\newcommand{\OtherTok}[1]{\textcolor[rgb]{0.00,0.44,0.13}{{#1}}}
\newcommand{\AlertTok}[1]{\textcolor[rgb]{1.00,0.00,0.00}{\textbf{{#1}}}}
\newcommand{\FunctionTok}[1]{\textcolor[rgb]{0.02,0.16,0.49}{{#1}}}
\newcommand{\RegionMarkerTok}[1]{{#1}}
\newcommand{\ErrorTok}[1]{\textcolor[rgb]{1.00,0.00,0.00}{\textbf{{#1}}}}
\newcommand{\NormalTok}[1]{{#1}}
\ifxetex
  \usepackage[setpagesize=false, % page size defined by xetex
              unicode=false, % unicode breaks when used with xetex
              xetex,
              colorlinks=true,
              linkcolor=blue]{hyperref}
\else
  \usepackage[unicode=true,
              colorlinks=true,
              linkcolor=blue]{hyperref}
\fi
\hypersetup{breaklinks=true, pdfborder={0 0 0}}
\setlength{\parindent}{0pt}
\setlength{\parskip}{6pt plus 2pt minus 1pt}
\setlength{\emergencystretch}{3em}  % prevent overfull lines
\setcounter{secnumdepth}{0}

%\EndDefineVerbatimEnvironment{Highlighting}

\title{Types with class}
\author{Wim Vanderbauwhede}
%\date{}                                           % Activate to display a given date or no date

\begin{document}
\maketitle
Type classes are a way to overload functions or operators by putting
constraints on polymorphism.

For example, we have seen that:

\begin{Shaded}
\begin{Highlighting}[]
\OtherTok{(+) ::} \NormalTok{a }\OtherTok{->} \NormalTok{a }\OtherTok{->} \NormalTok{a }
\end{Highlighting}
\end{Shaded}

is not OK because we want to restrict addition to numbers.

Likewise,

\begin{Shaded}
\begin{Highlighting}[]
\OtherTok{    (<) ::} \NormalTok{a }\OtherTok{->} \NormalTok{a }\OtherTok{->} \DataTypeTok{Bool}
\end{Highlighting}
\end{Shaded}

is not OK because it is not clear a priory how to compare to arbitrary
types.

To address this issue Haskell provides type classes. These restrict the
polymorphism. For example:

\begin{Shaded}
\begin{Highlighting}[]
\OtherTok{    (+) ::} \DataTypeTok{Num} \NormalTok{a }\OtherTok{=>} \NormalTok{a }\OtherTok{->} \NormalTok{a }\OtherTok{->} \NormalTok{a }
\end{Highlighting}
\end{Shaded}

says that the type a must be a numeric type, and

\begin{Shaded}
\begin{Highlighting}[]
\OtherTok{    (<) ::} \DataTypeTok{Ord} \NormalTok{a }\OtherTok{=>} \NormalTok{a }\OtherTok{->} \NormalTok{a }\OtherTok{->} \DataTypeTok{Bool}
\end{Highlighting}
\end{Shaded}

says that a must be orderable.
\end{document}

%-----------------------------------------------------------------------
% Functional Programming 4
% John O'Donnell, Wim Vanderbauwhede
% University of Glasgow
%-----------------------------------------------------------------------

\documentclass{beamer}
\usepackage{jtodlecseriesFP4}

%include polycode.fmt
%format alpha = "\alpha"
%format -- > = "\leadsto "

\newcommand{\alphaconv}{\hbox{\quad$\equiv\alpha$\quad}}
\newcommand{\betaconv}{\hbox{\quad$\equiv\beta$\quad}}
\newcommand{\etaconv}{\hbox{\quad$\equiv\eta$\quad}}

% Identify this presentation
\SetPresentationTitle
  {Monads}
  {Monads}
\SetPresentationNumber
  {12}
\SetPresentationDate
  {Week 6-2}
  {Week 6-2}

%-----------------------------------------------------------------------
% Beginning

\begin{document}

\begin{frame}[fragile]
  \PresentationTitleSlide
\end{frame}

\begin{frame}[fragile]
  \frametitle{Topics}
  \tableofcontents
\end{frame}
%-----------------------------------------------------------------------
\begin{frame}[fragile]
\begin{center}
\includegraphics[scale=0.075]
    {figures/jpg/pic09.jpg}
\end{center}
\end{frame}
%-----------------------------------------------------------------------
\section{Monads}

\begin{frame}[fragile]
\frametitle{Monads}

\begin{itemize}
\item A monad is an algebraic structure in category theory.
\item Monads are abstract, and they have many useful concrete
  instances.
\item Monads provide a way to structure a program.
\item They can be used (along with abstract data types) to allow
  actions (e.g. mutable variables) to be implemented safely.
\end{itemize}

\end{frame}

%-----------------------------------------------------------------------
\subsection{Building blocks}

\begin{frame}[fragile]
\frametitle{Building blocks of a monad}

A monad has three building blocks:

\begin{itemize}
\item A type constructor that produces the type of a computation,
  tiven the type of that computation's result.
\item A function that takes a value, and returns a computation
  that---when executed---will produce that value.
\item A function that takes two computations and performs them one
  after the other, making the result of the first computation
  available to the second.
\end{itemize}

This will be restated more precisely$\ldots$

\end{frame}


%-----------------------------------------------------------------------
\begin{frame}[fragile]
\frametitle{Definition}

A monad consists of three objects (below), which must satisfy the
monad laws (later).

\begin{itemize}
\item A type constructor $M$, such that for any type $a$, the type
  $M a$ is the type of a computation in the monad $M$ that produces
  a result of type $a$.
\item A function $return :: a \rightarrow M\, a$.  Thus if $x::a$, then
  $return \, x$ is a computation in $M$ that, when executed, will
  produce a value of type $a$.
\item A function $(>>=) :: M\, a \rightarrow  (a \rightarrow  M\, b) \rightarrow  M\, b$.
  \begin{itemize}
  \item The first argument is a computation that produces a value
    of type $a$.
  \item The second argument is a function that requires an argument
    of type $a$ and returns a computation that produces a value of
    type $b$.
  \item The result is a computation that will produce a value of
    type $b$.  It works by running the first computation and
    passing its result to the function that returns the second
    computation, which is then executed.
  \end{itemize}
\end{itemize}

\end{frame}

%-----------------------------------------------------------------------
\begin{frame}[fragile]
\frametitle{The definition says what a monad is!}

\begin{itemize}
\item There are many metaphors or intuitions about what a monad is.
\item Example: a ``computation'' or an ``action''.
\item But these terms are vague---they may help you to understand,
  but they can also be misleading at times.
\item \emph{A monad is exactly what the definition says, no more
    and no less.}
\end{itemize}

\end{frame}

%-----------------------------------------------------------------------
\subsection{Type class}

\begin{frame}[fragile]
\frametitle{Generality of monads}

\begin{itemize}
\item Monads are abstract and general and useful.
\item Consequently, there are many instances of them.
\item This is captured by defining a type class for monads, along
  with many standard instances.  And you can define your own.
\end{itemize}

\end{frame}

%-----------------------------------------------------------------------
\begin{frame}[fragile]
\frametitle{Monad type class}

\begin{verbatim}
class Monad m where
    return :: a -> m a
    (>>=) :: m a -> (a -> m b) -> m b
    (>>)   :: m a -> m b -> m b
    fail   :: String -> m a
\end{verbatim}

\begin{itemize}
\item The $fail$ function is used by the system to help produce
  error messages when there is a pattern that fails to match;
  normally you don't use it directly.
\item We'll just pretend $fail$ isn't there.
\end{itemize}

\end{frame}

%-----------------------------------------------------------------------
\begin{frame}[fragile]
\frametitle{The ``then'' operator}

\begin{itemize}
\item The ``bind'' operator $>>=$ is a crucial part of a monad.
\item Sometimes the value returned by a computation is unimportant.
%\item If you apply a function to a value but the function doesn't
%  need the value, you can use the underscore as a dummy variable
%  name: {\tt ( \_ -> 23) (2+2)}.
\item The $>>$ operator is like $>>=$, but it just ignores the
  value returned by the computation.
\end{itemize}

\begin{verbatim}
m >> n = m >>= \_ -> n
\end{verbatim}

\end{frame}

%-----------------------------------------------------------------------
\subsection{Laws}
\begin{frame}[fragile]
\frametitle{Monad laws}

Every monad must satisfy the following three laws.

\vspace{1em}

The \emph{right unit law}:

\begin{verbatim}
m >>= return = m
\end{verbatim}

The \emph{left unit law}:

\begin{verbatim}
return x >>= f  = f x
\end{verbatim}

The \emph{associativity law}:

\begin{verbatim}
(m >>= f) >>= g = m >>= (\x -> f x >>= g)
\end{verbatim}

\end{frame}

%-----------------------------------------------------------------------
\section{The Maybe Monad}
\begin{frame}[fragile]
\frametitle{The Maybe Monad}

\begin{itemize}
\item We'll look at several examples, of monads, starting with Maybe.
\end{itemize}

\end{frame}

%-----------------------------------------------------------------------
\subsection{The $Maybe$ type constructor}
\begin{frame}[fragile]
\frametitle{The $Maybe$ type constructor}

\begin{verbatim}
import Control.Monad
\end{verbatim}

\begin{verbatim}
data MyMaybe a = MyNothing | MyJust a
\end{verbatim}

\end{frame}

%-----------------------------------------------------------------------
\begin{frame}[fragile]
\frametitle{Safe head and tail}

\begin{verbatim}
myHead :: [a] -> Maybe a
myHead [] = Nothing
myHead (x:xs) = Just x
\end{verbatim}

\begin{verbatim}
myTail :: [a] -> Maybe [a]
myTail [] = Nothing
myTail (x:xs) = Just xs
\end{verbatim}

\end{frame}

%-----------------------------------------------------------------------
\subsection{Monad instance of Maybe}
\begin{frame}[fragile]
\frametitle{Monad instance of Maybe}

\begin{verbatim}
instance Monad MyMaybe where
    return           =   MyJust
    MyNothing  >>= f = MyNothing
    (MyJust x) >>= f = f x
    fail _           =   MyNothing
\end{verbatim}

    
\begin{verbatim}
instance MonadPlus MyMaybe where
    mzero               = MyNothing
    MyNothing `mplus` x = x
    x `mplus` _         = x
\end{verbatim}

\end{frame}

%-----------------------------------------------------------------------
\subsection{A computation using explicit Maybe}
\begin{frame}[fragile]
\frametitle{A computation using explicit Maybe}

\begin{verbatim}
foo :: [a] -> Maybe a
foo xs =
  case myTail xs of
    Nothing -> Nothing
    Just a -> case myTail a of
                Nothing -> Nothing
                Just b -> myHead b
\end{verbatim}

\end{frame}

%-----------------------------------------------------------------------
\subsection{A computation using the Maybe monad}
\begin{frame}[fragile]
\frametitle{A computation using the Maybe monad}

\begin{verbatim}
bar :: [a] -> Maybe a
bar xs =
  myTail xs >>=
    (\a -> myTail a >>=
      (\b -> myHead b))
\end{verbatim}

\end{frame}

%-----------------------------------------------------------------------
\begin{frame}[fragile]
\frametitle{Syntax: Changing the line breaks and indentation}

\begin{verbatim}
bar2 :: [a] -> Maybe a
bar2 xs =
  myTail xs >>= (\a ->
  myTail a >>=  (\b ->
  myHead b))
\end{verbatim}

\end{frame}

%-----------------------------------------------------------------------
\begin{frame}[fragile]
\frametitle{Syntax: Removing unnecessary parentheses}

\begin{verbatim}
bar3 :: [a] -> Maybe a
bar3 xs =
  myTail xs >>= \a ->
  myTail a >>=  \b ->
  myHead b
\end{verbatim}

\end{frame}

%-----------------------------------------------------------------------
\subsection{Example: Reduction of {\tt bar [5,6]}}
\begin{frame}[fragile]
\frametitle{Example: Reduction of bar [5,6]}

\begin{verbatim}
    bar [5,6]

-- > substitute [5,6] for xs in definition of bar

    myTail [5,6] >>=
     (\a -> myTail a >>=
      (\b -> myHead b))

-- > def. myTail

    myJust [6]  >>=
     (\a -> myTail a >>=
      (\b -> myHead b))

-- >  def.2 of (>>=)

     (\a -> myTail a >>=
      (\b -> myHead b))
        [6]
\end{verbatim}

\end{frame}

%-----------------------------------------------------------------------
\begin{frame}[fragile]
\frametitle{Reduction of bar [5,6] continued}

\begin{verbatim}
-- > beta reduction, substitute [6] for a

     myTail [6] >>= (\b -> myHead b)

-- > reduce myTail

     Just [] >>=  (\b -> myHead b)

-- >  def.2 of (>>=)   

    (\b -> myHead b) []

-- > beta reduction, substitute [] for b

   myHead []

-- > def.1 myHead

  Nothing
\end{verbatim}

\end{frame}

%-----------------------------------------------------------------------
\subsection{Example: Reduction of {\tt bar [5,6,7]}}
\begin{frame}[fragile]
\frametitle{Example: Reduction of bar [5,6,7]}

\begin{verbatim}
    bar [5,6,7]

-- > substitute [5,6,7] for xs in definition of bar

    myTail [5,6,7] >>=
     (\a -> myTail a >>=
      (\b -> myHead b))

-- > def. myTail

    myJust [6,7]  >>=
     (\a -> myTail a >>=
      (\b -> myHead b))

-- >  def.2 of (>>=)

     (\a -> myTail a >>=
      (\b -> myHead b))
        [6,7]
\end{verbatim}

\end{frame}

%-----------------------------------------------------------------------
\begin{frame}[fragile]
\frametitle{Reduction of bar [5,6,7] continued}

\begin{verbatim}
-- > beta reduction, substitute [6,7] for a

     myTail [6,7] >>= (\b -> myHead b)

-- > reduce myTail

     Just [7] >>=  (\b -> myHead b)

-- >  def.2 of (>>=)   

    (\b -> myHead b) [7]

-- > beta reduction, substitute [7] for b

    myHead [7]

-- > def myHead

    Just 7
\end{verbatim}

\end{frame}

%-----------------------------------------------------------------------
\subsection{{\tt do} notation}
\begin{frame}[fragile]
\frametitle{do notation}

\begin{verbatim}
baz :: [a] -> Maybe a

baz xs =
  do  a <- myTail xs
      b <- myTail a
      c <- myHead b
      return c
\end{verbatim}

\end{frame}

%-----------------------------------------------------------------------
\begin{frame}[fragile]
\frametitle{Syntax rules for do}

\begin{verbatim}
do { x }  -- >  x
\end{verbatim}

\begin{verbatim}
do {x ; <xs> }  -- >  x >> do { <xs> }
\end{verbatim}

\begin{verbatim}
do { a <- x ; <xs> }  -- >  x >>= \a -> do { <xs> }
\end{verbatim}

\begin{verbatim}
do { let <declarations> ; xs }
  -- >
let <declarations> in do { xs }
\end{verbatim}


\end{frame}
   
\end{document}

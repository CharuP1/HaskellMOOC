%-----------------------------------------------------------------------
% Functional Programming 4
% John O'Donnell, Wim Vanderbauwhede
% University of Glasgow
%-----------------------------------------------------------------------

\documentclass{beamer}
\usepackage{jtodlecseriesFP4}

%include polycode.fmt
%format alpha = "\alpha"
%format ~> = "\leadsto "

\newcommand{\alphaconv}{\hbox{\quad$\equiv\alpha$\quad}}
\newcommand{\betaconv}{\hbox{\quad$\equiv\beta$\quad}}
\newcommand{\etaconv}{\hbox{\quad$\equiv\eta$\quad}}

% Identify this presentation
\SetPresentationTitle
  {Type classes and Lambda Calculus}
  {Type classes and Lambda Calculus}
\SetPresentationNumber
  {11}
\SetPresentationDate
  {Week 6-1}
  {Week 6-1}

%-----------------------------------------------------------------------
% Beginning

\begin{document}

\begin{frame}[fragile]
  \PresentationTitleSlide
\end{frame}
%-----------------------------------------------------------------------
\begin{frame}
\begin{center}
\includegraphics[scale=0.2]
    {figures/jpg/pic07b.jpg}
\end{center}
\end{frame}
\begin{frame}[fragile]
  \frametitle{Topics}
  \tableofcontents
\end{frame}


%-----------------------------------------------------------------------
\section{Defining type classes}

\subsection{Class and instance declarations}

\begin{frame}[fragile]
\frametitle{Defining type classes}

\begin{itemize}
\item A \emph{type class} is a set of types for which some
  operations are defined.
\item Haskell has some standard type classes that are defined in
  the Standard Prelude.
\item You can also define your own.
\end{itemize}

\end{frame}

%-----------------------------------------------------------------------
\begin{frame}[fragile]
\frametitle{A type for bright colors}

Suppose we're computing with colors.  Here's a type, and a couple
of functions.

\begin{verbatim}
data Bright
  = Blue
  | Red
  deriving (Read, Show)
\end{verbatim}

\begin{verbatim}
darkBright :: Bright -> Bool
darkBright Blue = True
darkBright Red  = False
\end{verbatim}

\begin{verbatim}
lightenBright :: Bright -> Bright
lightenBright Blue = Red
lightenBright Red = Red
\end{verbatim}

\end{frame}

%-----------------------------------------------------------------------
\begin{frame}[fragile]
\frametitle{A type for milder colors}

Now, suppose we have a different type that needs similar functions.

\begin{verbatim}
data Pastel
  = Turquoise
  | Tan
  deriving (Read, Show)
\end{verbatim}

\begin{verbatim}
darkPastel :: Pastel -> Bool
darkPastel Turquoise = True
darkPastel Tan       = False
\end{verbatim}

\begin{verbatim}
lightenPastel :: Pastel -> Pastel
lightenPastel Turquoise = Tan
lightenPastel Tan       = Tan
\end{verbatim}

\end{frame}

%-----------------------------------------------------------------------
\begin{frame}[fragile]
\frametitle{Defining a type class}

\begin{itemize}
\item Both of our color types have functions to decide whether it's
  dark, or to lighten it.
\item We can define a class $Color$ and its corresponding
  functions.
\end{itemize}

\begin{verbatim}
class Color a where
  dark :: a -> Bool
  lighten :: a -> a
\end{verbatim}

This says
\begin{itemize}
\item $Color$ is a type class
\item The type variable $a$ stands for a particular type that is in
  the class $Color$
\item For any type $a$ in $Color$, there are two functions you can
  use: $dark$ and $lighten$, with the specified types.
\end{itemize}

\end{frame}

%-----------------------------------------------------------------------
\begin{frame}[fragile]
\frametitle{Defining instances for the type class}

\begin{itemize}
\item An $instance$ declaration says that a type is a member of a
  type class.
\item When you declare an instance, you need to define the class
  functions.
\item The following says that the type $Bright$ is in the class
  $Color$, and for that instance, the $dark$ function is actually
  $darkBright$.
\end{itemize}

\begin{verbatim}
instance Color Bright where
  dark = darkBright
  lighten = lightenBright
\end{verbatim}

\begin{itemize}
\item Similarly, we can declare that $Pastel$ is in $Color$, but it
  has different functions to implement the class operations.
\end{itemize}

\begin{verbatim}
instance Color Pastel where
  dark = darkPastel
  lighten = lightenPastel
\end{verbatim}

\end{frame}

%-----------------------------------------------------------------------
\subsection{The Num class}

\begin{frame}[fragile]
\frametitle{The Num class}

\begin{itemize}
\item Haskell provides several standard type classes.
\item $Num$ is the class of numeric types.
\item Here is (part of) its class declaration:
\end{itemize}

\begin{verbatim}
class Num a where
  (+), (-), (*) :: a -> a -> a
\end{verbatim}

\end{frame}

%-----------------------------------------------------------------------
\begin{frame}[fragile]
\frametitle{Num instances}

\begin{itemize}
\item There are many numeric types; two of them are $Int$ and
  $Double$.
\item There are primitive monomorphic functions that perform
  arithmetic on these types (these aren't the real names):
  \begin{itemize}
  \item $addInt, subInt, MulInt :: Int -> Int -> Int$
  \item $addDbl, subDbl, MulDbl :: Double -> Double -> Double$
  \end{itemize}
\end{itemize}

\begin{verbatim}
instance Num Int where
  (+) = addInt
  (-) = subInt
  (*) = mulInt

instance Num Double where
  (+) = addDbl
  (-) = subDbl
  (*) = mulDbl
\end{verbatim}

\end{frame}

%-----------------------------------------------------------------------
\begin{frame}[fragile]
\frametitle{Hierarchy of numeric classes}

\begin{itemize}
\item There are some operations (addition) that are valid for all
  numeric types.
\item There are some others (e.g. trigonometric functions) that are
  valid only for \emph{some} numeric types.
\item Therefore there is a rich hierarchy of subclasses, including
  \begin{itemize}
  \item $Integral$ --- class of numeric types that represent
    integer values, including $Int$, $Integer$, and more.
  \item $Fractional$ --- class of types that can represent fractions.
  \item $Floating$ --- class containing $Float$, $Double$, etc.
  \item $Bounded$ --- class of numeric types that have a minimal
    and maximal element.
  \item $Bits$ --- class of types where you can access the
    representation as a sequence of bits, useful for systems
    programming and digital circuit design.
  \end{itemize}
\item If you want to get deeply into numeric classes and types,
  refer to the documentation.
\end{itemize}

\end{frame}

%-----------------------------------------------------------------------
\subsection{The $Show$ class}

\begin{frame}[fragile]
\frametitle{The $Show$ class}

\begin{itemize}
\item We have been using $show$ to convert a data value to a
  string, which can then be written to output.
\item Some values can be ``shown'', but not all.
\item For example, it is impossible in general to show a function.
\item Therefore $show$ needs a type class!
\item $show :: Show\, a \Rightarrow a \rightarrow \,String$
\end{itemize}

\end{frame}

%-----------------------------------------------------------------------
\begin{frame}[fragile]
\frametitle{Defining your own Show instance}

\begin{verbatim}
data Foo = Bar | Baz
\end{verbatim}

We might want our own peculiar string representation:

\begin{verbatim}
instance Show Foo where
  show Bar = "it is a bar"
  show Baz = "this is a baz"
\end{verbatim}

Recall that when you enter an expression $exp$ into ghci, it prints
$show exp$.  So we can try out our strange instance declaration:

\begin{verbatim}
*Main> Bar
it is a bar
*Main> Baz
this is a baz
\end{verbatim}

\end{frame}

%-----------------------------------------------------------------------
\begin{frame}[fragile]
\frametitle{Deriving Show}

This is a similar type, but it has a $deriving$ clause.

\begin{verbatim}
data Foo2 = Bar2 | Baz2
  deriving (Read, Show)
\end{verbatim}

Haskell will automatically define an instance of $show$ for $Foo2$,
using the obvious definition:

\begin{verbatim}
*Main> Bar2
Bar2
*Main> Baz2
Baz2
\end{verbatim}

\end{frame}

%-----------------------------------------------------------------------
\subsection{More standard typeclasses}

\begin{frame}[fragile]
\frametitle{More standard typeclasses}

Here is a summary of some of the type classes defined in the
standard libraries.

\begin{itemize}
\item $Num$ --- numbers, with many subclasses for specific kinds of number.
\item $Read$ --- types that can be ``read in from'' a string.
\item $Show$ --- types that can be ``shown to'' a string.
\item $Eq$ --- types for which the equality operator $==$ is defined.
\item $Ord$ --- types for which you can do comparisons like $<$,
  $>$, etc.
\item $Enum$ --- types where the values can be enumerated in
  sequence; this is used for example in the notation $[1..n]$ and
  $'a'..'z'$.
\end{itemize}

\begin{verbatim}
*Main> [1..10]
[1,2,3,4,5,6,7,8,9,10]
*Main> ['a'..'z']
"abcdefghijklmnopqrstuvwxyz"
\end{verbatim}

\end{frame}

%-----------------------------------------------------------------------
\section{Introduction to Lambda Calculus}

\begin{frame}[fragile]
\frametitle{Introduction to Lambda Calculus}

\begin{itemize}
\item The lambda calculus was developed in the 1930s by Alonzo
  Church (1903--1995), one of the leading developers of
  mathematical logic.
\item The lambda calculus was an attempt to formalise functions as
  a means of computing.
\end{itemize}

\end{frame}

%-----------------------------------------------------------------------
\begin{frame}[fragile]
\frametitle{Significance to computability theory}

\begin{itemize}
\item A major (really \emph{the} major) breakthrough in
  computability theory was the proof that the lambda calculus and
  the Turing machine have exactly the same computational power.
\item This led to \emph{Church's thesis} --- that the set of
  functions that are effectively computable are exactly the set
  computable by the Turing machine or the lambda calculus.
\item The thesis was strengthened when several other mathematical
  computing systems (Post Correspondence Problem, and others) were
  also proved equivalent to lambda calculus.
\item The point is that the set of effectively computable functions
  seems to be a fundamental reality, not just a quirk of how the
  \{Turing machine, lambda calculus\} was defined.
\end{itemize}

\end{frame}

%-----------------------------------------------------------------------
\begin{frame}[fragile]
\frametitle{Significance to programming languages}

\begin{itemize}
\item The lambda calculus has turned out to capture two aspects of
  a function:
  \begin{itemize}
  \item A mathematical object (set or ordered pairs from domain and
    range), and
  \item An abstract black box machine that takes an input and
    produces an output.
  \end{itemize}
\item The lambda calculus is fundamental to denotational semantics,
  the mathematical theory of what computer programs mean.
\item Functional programming languages were developed with the
  explicit goal of turning lambda calculus into a practical
  programming language.
\item The ghc Haskell compiler operates by (1) desugaring the
  source program, (2) transforming the program into a version of
  lambda calculus called \emph{System F}, and (3) translating the
  System F to machine language using \emph{graph reduction}.
\end{itemize}

\end{frame}

%-----------------------------------------------------------------------
\begin{frame}[fragile]
\frametitle{Abstract syntax of lambda calculus}

\begin{itemize}
\item We will work with the basic lambda calculus ``enriched'' with
  some constants and primitive functions (strictly speaking, that
  is not necessary).
\item The language has constants, variables, applications, and
  functions.
\end{itemize}

\begin{verbatim}
exp
  = const
  | var
  | exp exp
  | \ var -> exp
\end{verbatim}

\end{frame}

%-----------------------------------------------------------------------
\begin{frame}[fragile]
\frametitle{Variables}

\begin{itemize}
\item Each occurrence of a variable in an expression is either
  \emph{bound} or \emph{free}
  \begin{itemize}
  \item In $\backslash x \rightarrow x+1$, the occurrence of $x$ in $x+1$ is
    \emph{bound} by the $\backslash x$.
  \item In $y*3$, the occurrence or $y$ is \emph{free}.  It must be
    defined somewhere else, perhaps as a global definition.
  \end{itemize}
\item In general, an occurrence of a variable is bound if there is
  some enclosing lambda expression that binds it; if there is no
  lambda binding, then the occurrence if free.
\end{itemize}

We need to be careful: the first occurrence of $a$ is free but the
second occurrence is bound.

\begin{verbatim}
   a + (\ a -> 2^a) 3  -- >   a + 2^3
\end{verbatim}

Being free or bound is a property of an \emph{occurrence} of a
variable, not of the variable itself!

\end{frame}

%-----------------------------------------------------------------------
\begin{frame}[fragile]
\frametitle{Conversion rules}

\begin{itemize}
\item Computing in the lambda calculus is performed using three
  \emph{conversion rules}.
\item The conversion rules allow you to replace an expression by
  another (``equal'') one.
\item Some conversions simplify an expression; these are called
  \emph{reductions}.
\end{itemize}

\end{frame}

%-----------------------------------------------------------------------
\begin{frame}[fragile]
\frametitle{Alpha conversion}

\begin{itemize}
\item Alpha conversion lets you change the name of a function
  parameter consistently.
\item But you can't change free variables with alpha conversion!
\item The detailed definition of alpha conversion is a bit tricky,
  because you have to be careful to be consistent and avoid ``name
  capture''.  We won't worry about the details right now.
\end{itemize}

$(\backslash x \rightarrow x+1) \,3$ \alphaconv $(\backslash y \rightarrow y+1) \,3$

\end{frame}

%-----------------------------------------------------------------------
\begin{frame}[fragile]
\frametitle{Beta conversion}

\begin{itemize}
\item Beta conversion is the ``workhorse'' of lambda calculus: it
  defines how functions work.
\item To apply a lambda expression an argument, you take the body
  of the function, and replace each bound occurrence of the
  variable with the argument.
\end{itemize}

$(\backslash x \rightarrow exp1) \;exp2$ \betaconv $exp1[exp2/x]$

\vspace{2em}
Example:
\vspace{0.5em}

$(\backslash x \rightarrow 2*x + g\; x) \;42$ \betaconv $2*42 + g \;42$

\end{frame}

%-----------------------------------------------------------------------
\begin{frame}[fragile]
\frametitle{Eta conversion}

\begin{itemize}
\item Eta conversion says that a function is equivalent to a lambda
  expression that takes an argument and applies the function to the
  argument.
\end{itemize}

$(\backslash x -> f x)$ \etaconv $f$

\vspace{2em}

Example (recall that $(*3)$ is a function that multiplies its
argument by 3)
\vspace{0.5em}

$(\backslash x \rightarrow (*3) \;x)$ \etaconv $(*3)$

\vspace{0.5em}
Try applying both of these to 50:
\vspace{0.5em}
$(\backslash x \rightarrow (*3) \;x) \;50\rightsquigarrow (*3) \; 50$ which is the same as $(*3) \;50$

\end{frame}

%-----------------------------------------------------------------------
\begin{frame}[fragile]
\frametitle{Removing a common trailing argument}

There is a common usage of Eta conversion.  Suppose we have a
definition like this:

$f \;x \;y = g\; y$

\vspace{1em}

This can be rewritten as follows:

\vspace{1em}

$f = \backslash x \rightarrow (\backslash y \rightarrow g \;y)$
\etaconv
$f = \backslash x \rightarrow g$
\quad $\equiv$ \quad
$f\, x = g$

Thus the following two definitions are equivalent:

\begin{verbatim}
f x y = g y
f x = g
\end{verbatim}

In effect, since the last argument on both sides of the equation is
the same ($y$), it can be ``factored out''.

\end{frame}
%-----------------------------------------------------------------------
%\section{A note on file I/O}
%
%\begin{frame}[fragile]
%\frametitle{A note on file I/O}
%
%\begin{itemize}
%\item There are many library computations to support input/output.
%\item A simple way to read a file, suitable for the assessed
%  exercise, is $readFile$.
%\end{itemize}
%
%\begin{verbatim}
%readFile :: FilePath -> IO String
%\end{verbatim}
%
%A $FilePath$ is just a synonym for a $String$.
%
%\begin{verbatim}
%prog :: IO ()
%prog =
%  do putStrLn "hello"
%     xs <- readFile "Foo.txt"
%     putStr xs
%     return ()
%\end{verbatim}
%
%\end{frame}

\end{document}

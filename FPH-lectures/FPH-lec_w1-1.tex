%-----------------------------------------------------------------------
% Functional Programming 4
% John O'Donnell, Wim Vanderbauwhede
%-----------------------------------------------------------------------

\documentclass{beamer}
%include polycode.fmt
%format alpha = "\alpha"
%format ~> = "\leadsto "
\usepackage{jtodlecseriesFP4}
\usepackage{url}
% Identify this presentation
\SetPresentationTitle
  {Getting Started}
  {Getting Started}
\SetPresentationNumber
  {1}
\SetPresentationDate
  {Week 1-1}
  {Week 1-1}

%-----------------------------------------------------------------------
% Beginning

\begin{document}

\begin{frame}
  \PresentationTitleSlide
\end{frame}

\begin{frame}
 \frametitle{Topics}
 \tableofcontents
\end{frame}

%-----------------------------------------------------------------------
\section{Expressions}

\begin{frame}
\frametitle{Expressions in Haskell}

\begin{itemize}
\item In an imperative language like C or Java,
  \begin{itemize}
  \item there are expressions that denote small scale computations
    (2*x), and
  \item  \emph{statements} that handle sequencing, looping,
    conditionals, and all the large scale operation of the program.
  \end{itemize}
\item \emph{Pure functional programming languages don't have any statements} --- no
  assignments, no jumps.
\item Instead, \emph{all computation is performed by evaluating
    expressions}
\item So, let's start with expressions!
  \begin{itemize}
  \item (We'll still be working on expressions at the end of the
    course, since that's all there is.)
  \end{itemize}
\end{itemize}

\end{frame}
%-----------------------------------------------------------------------

\begin{frame}[fragile]
\frametitle{Examples of integer expressions}

An expression evaluates to a result (usually written $e \rightsquigarrow r$ but we'll use \texttt{e -- > r}).  Haskell uses a similar notation for numbers and operators as most languages:

\begin{verbatim}
2 -- > 2
\end{verbatim}

\begin{verbatim}
3+4 -- > 7
\end{verbatim}

\begin{verbatim}
3+4*5    {equivalent to 3+(4*5)} -- > 23
\end{verbatim}

\begin{verbatim}
(3+4)*5   {equivalent to 7*5} -- > 35    
\end{verbatim}

\end{frame}

%-----------------------------------------------------------------------
\begin{frame}
\frametitle{Syntax of expressions}

\begin{itemize}
\item Parentheses are used for grouping, just as in mathematics. 
\item If you don't need parentheses for grouping, they are
  optional.
\item Operators have precedence, e.g. $*$ has ``tighter''
  precedence than $+$, so $2+3*4$ means $2+(3*4)$.
\item Use reference documentation for complete list of operators
  and their precedences, if you need them.
%\item Warning!  Don't write just -3 to get the number "negative
 % three", you often need to write (-3).
\end{itemize}
\end{frame}

\begin{frame}[fragile]
\frametitle{Function applications}

\begin{itemize}
\item A function takes argument(s), performs some computation, and
  produces result(s).
\item The function $abs$ gives the absolute value of a number.
\item To use a function, you apply it to an argument.  Write the
  function followed by the argument, separated by a space.
\end{itemize}

\begin{verbatim}
abs 5 -- > 5
\end{verbatim}

\begin{verbatim}
abs (-6) -- > 6
\end{verbatim}

\end{frame}

%-----------------------------------------------------------------------
\begin{frame}[fragile]

\frametitle{Parentheses are for grouping}

Good style

\begin{verbatim}
2+3*5
2+(3*5)     {might be clearer to some readers}
abs 7
\end{verbatim}

You don't need parentheses.  The following are legal, but they look
silly:

\begin{verbatim}
(2) + ((3+(((((5)))))))
abs (5)
abs (((5)))
\end{verbatim}

\end{frame}

%-----------------------------------------------------------------------

%-----------------------------------------------------------------------
\begin{frame}[fragile]
\frametitle{Functions with several arguments}

\begin{itemize}
\item $min$ and $max$ are functions that take two arguments.
\item The arguments are given after the function,  separated by
  whitespace. 
\item Write $min\; 3\; 8$, don't write $min(3,8);$
\end{itemize}

\begin{verbatim}
  min 3 8
-- >
  3

  max 3  8
-- >
  8
\end{verbatim}

\end{frame}

%-----------------------------------------------------------------------
\begin{frame}
\frametitle{Precedence of function application}

\begin{itemize}
\item Function application binds tighter than anything else.
\item So \texttt{f x + 3} means \texttt{(f x) + 3} and not \texttt{f (x+3)}
\item If an argument to a function is an expression, you'll need to
  put it in parentheses.
\end{itemize}

\end{frame}

%-----------------------------------------------------------------------
\section{Equations}

\begin{frame}[fragile]
\frametitle{Equations}

\begin{itemize}
\item Equations are used to give names to values.
\end{itemize}

\begin{verbatim}
answer = 42
\end{verbatim}
\end{frame}

%-----------------------------------------------------------------------
\begin{frame}
\frametitle{Equations give names to values}

\begin{itemize}
\item An equation in Haskell is a mathematical equation: it says
  that the left hand side and the right hand side denote the same
  value.
\item The left hand side should be a name that you're giving a
  value to.
  \begin{itemize}
  \item Correct: \texttt{x = 5*y}
  \item Incorrect: \texttt{2 * x = (3*x)**2} -- Reassignment is not allowed in a pure FPL
  \end{itemize}
\end{itemize}

\end{frame}

%-----------------------------------------------------------------------
\begin{frame}[fragile]
\frametitle{Equations are not assignments}

\begin{itemize}
\item A name \emph{can be given only one value}.
\item Names are often called ``variables'', but they \emph{do not
    vary}.
\item In Haskell \emph{variables are constant}!
 
\end{itemize}

\begin{verbatim}
n = 1    -- just fine!
x = 3*n  -- fine
n = x    -- Wrong: can have only one definition of n
\end{verbatim}

\begin{itemize}
\item Once you give a value to a name, you can never change it!
\item This is part of the meaning of ``pure'' and
  ``no side effects''
\end{itemize}

\end{frame}

%-----------------------------------------------------------------------

\begin{frame}
\frametitle{What about n = n+1?}

\begin{itemize}
\item In imperative languages, we frequently say $n := n+1$.
  \begin{itemize}
  \item This is an assignment, not an equation!
  \item It means (1) compute the right hand side, using the old
    value of $n$; then (2) discard the old value of $n$ and
    overwrite it with the new value.
  \item \emph{There are no equations in imperative languages like C
      and Java.}
  \end{itemize}
\item In Haskell, it is valid to write $n = n+1$.
  \begin{itemize}
  \item This is an equation, not an assignment!
  \item It means: compute the value of $n$ that has the property
    that $n = n+1$.
  \item Haskell will try, and it will fail.
\end{itemize}
\end{itemize}

\end{frame}




%-----------------------------------------------------------------------
\begin{frame}
\frametitle{How can you compute without assignments?}

\begin{itemize}
\item Think of an assignment statement as doing three things:
  \begin{enumerate}
  \item It evaluates the right hand side: computing a useful value.
  \item It discards the value of the variable on the left hand
    side: destroying a value that might or might not be useful.
  \item It saves the useful value of the RHS into the variable.
  \end{enumerate}
\item In a pure functional language
  \begin{itemize}
  \item We never destroy old values.
  \item We just compute new useful ones.
  \item If the old value was truly useless, the garbage collector
    will reclaim its storage.
  \end{itemize}
\end{itemize}

\end{frame}

%-----------------------------------------------------------------------
\section{The Haskell interpreter ghci}



%-----------------------------------------------------------------------
\begin{frame}[fragile]
\frametitle{Launching ghci}

To launch the  Haskell interpreter \emph{ghci}:
{\footnotesize
\begin{verbatim}
[wim@fp4 ~]$ghci
GHCi, version 7.8.3: http://www.haskell.org/ghc/  :? for help
Loading package ghc-prim ... linking ... done.
Loading package integer-gmp ... linking ... done.
Loading package base ... linking ... done.
Prelude> 
\end{verbatim}
}

Evaluate an expression

\begin{verbatim}
ls> 3+4
7
\end{verbatim}

\end{frame}

%-----------------------------------------------------------------------
\begin{frame}
\frametitle{Try Haskell!}

\begin{itemize}
\item You can install the Haskell compiler/interpreter on your own computer. Go to \url{https://www.haskell.org/platform} to get the Haskell Platform for your system, it is very easy to install.

\item \emph{All software used in this course is free software.}
\item Try experimenting with the expressions shown in this lecture.
\item And try some experiments of your own.
\end{itemize}

\end{frame}






\end{document}

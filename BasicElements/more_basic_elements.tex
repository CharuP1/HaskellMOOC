\documentclass[11pt]{amsart}
\usepackage{geometry}                % See geometry.pdf to learn the layout options. There are lots.
\geometry{letterpaper}                   % ... or a4paper or a5paper or ... 
%\geometry{landscape}                % Activate for for rotated page geometry
%\usepackage[parfill]{parskip}    % Activate to begin paragraphs with an empty line rather than an indent
%\usepackage{hyperref}
\usepackage{graphicx}
\usepackage{amssymb}
\usepackage{epstopdf}
\DeclareGraphicsRule{.tif}{png}{.png}{`convert #1 `dirname #1`/`basename #1 .tif`.png}
% From https://gist.github.com/cormacrelf/3760427
\usepackage{color}
\usepackage{fancyvrb}
\DefineShortVerb[commandchars=\\\{\}]{\|}
\DefineVerbatimEnvironment{Highlighting}{Verbatim}{commandchars=\\\{\}}
% Add ',fontsize=\small' for more characters per line
\newenvironment{Shaded}{}{}
\newcommand{\KeywordTok}[1]{\textcolor[rgb]{0.00,0.44,0.13}{\textbf{{#1}}}}
\newcommand{\DataTypeTok}[1]{\textcolor[rgb]{0.56,0.13,0.00}{{#1}}}
\newcommand{\DecValTok}[1]{\textcolor[rgb]{0.25,0.63,0.44}{{#1}}}
\newcommand{\BaseNTok}[1]{\textcolor[rgb]{0.25,0.63,0.44}{{#1}}}
\newcommand{\FloatTok}[1]{\textcolor[rgb]{0.25,0.63,0.44}{{#1}}}
\newcommand{\CharTok}[1]{\textcolor[rgb]{0.25,0.44,0.63}{{#1}}}
\newcommand{\StringTok}[1]{\textcolor[rgb]{0.25,0.44,0.63}{{#1}}}
\newcommand{\CommentTok}[1]{\textcolor[rgb]{0.38,0.63,0.69}{\textit{{#1}}}}
\newcommand{\OtherTok}[1]{\textcolor[rgb]{0.00,0.44,0.13}{{#1}}}
\newcommand{\AlertTok}[1]{\textcolor[rgb]{1.00,0.00,0.00}{\textbf{{#1}}}}
\newcommand{\FunctionTok}[1]{\textcolor[rgb]{0.02,0.16,0.49}{{#1}}}
\newcommand{\RegionMarkerTok}[1]{{#1}}
\newcommand{\ErrorTok}[1]{\textcolor[rgb]{1.00,0.00,0.00}{\textbf{{#1}}}}
\newcommand{\NormalTok}[1]{{#1}}
\ifxetex
  \usepackage[setpagesize=false, % page size defined by xetex
              unicode=false, % unicode breaks when used with xetex
              xetex,
              colorlinks=true,
              linkcolor=blue]{hyperref}
\else
  \usepackage[unicode=true,
              colorlinks=true,
              linkcolor=blue]{hyperref}
\fi
\hypersetup{breaklinks=true, pdfborder={0 0 0}}
\setlength{\parindent}{0pt}
\setlength{\parskip}{6pt plus 2pt minus 1pt}
\setlength{\emergencystretch}{3em}  % prevent overfull lines
\setcounter{secnumdepth}{0}

%\EndDefineVerbatimEnvironment{Highlighting}




\title{More elements of Haskell by example}
\author{Wim Vanderbauwhede}
%\date{}                                           % Activate to display a given date or no date

\begin{document}
\maketitle
%\section{}
%\subsection{}



A few more basic (and not-so-basic) elements of Haskell through
comparison with other languages. We will not go into detail on the
\texttt{Haskell} constructs, just show the similarities with constructs
from languages you may know.

\subsection{Blocks}\label{blocks}

In \texttt{JavaScript} functions typically are blocks of code:

\begin{Shaded}
\begin{Highlighting}[]
    \KeywordTok{function} \FunctionTok{roots}\NormalTok{(a,b,c) \{}
        \NormalTok{det2 = b*b}\DecValTok{-4}\NormalTok{*a*c;}
        \NormalTok{det  = }\FunctionTok{sqrt}\NormalTok{(det);}
        \NormalTok{rootp = (-b + det)/a/}\DecValTok{2}\NormalTok{;}
        \NormalTok{rootm = (-b - det)/a/}\DecValTok{2}\NormalTok{;}
        \KeywordTok{return} \NormalTok{[rootm,rootp]}
    \NormalTok{\}}
\end{Highlighting}
\end{Shaded}

In \texttt{Haskell}, we would write this function as follows:

\begin{Shaded}
\begin{Highlighting}[]
    \NormalTok{roots a b c }\FunctionTok{=} 
        \KeywordTok{let}
            \NormalTok{det2 }\FunctionTok{=} \NormalTok{b}\FunctionTok{*}\NormalTok{b}\FunctionTok{-}\DecValTok{4}\FunctionTok{*}\NormalTok{a}\FunctionTok{*}\NormalTok{c;}
            \NormalTok{det  }\FunctionTok{=} \NormalTok{sqrt(det);}
            \NormalTok{rootp }\FunctionTok{=} \NormalTok{(}\FunctionTok{-}\NormalTok{b }\FunctionTok{+} \NormalTok{det)}\FunctionTok{/}\NormalTok{a}\FunctionTok{/}\DecValTok{2}\NormalTok{;}
            \NormalTok{rootm }\FunctionTok{=} \NormalTok{(}\FunctionTok{-}\NormalTok{b }\FunctionTok{-} \NormalTok{det)}\FunctionTok{/}\NormalTok{a}\FunctionTok{/}\DecValTok{2}\NormalTok{;}
        \KeywordTok{in}
            \NormalTok{[rootm,rootp]}
\end{Highlighting}
\end{Shaded}

Note that the \texttt{let ... in ...} construct is an \emph{expression},
so it returns a value. That's why there is no need for a \texttt{return}
keyword.

\subsection{Conditions}\label{conditions}

In \texttt{Python} we could write a function with a condition as like
this:

\begin{Shaded}
\begin{Highlighting}[]
\KeywordTok{def} \DataTypeTok{max}\NormalTok{(x,y):}
    \KeywordTok{if} \NormalTok{x > y:}
        \KeywordTok{return} \NormalTok{x}
    \KeywordTok{else}\NormalTok{:}
        \KeywordTok{return} \NormalTok{y}
\end{Highlighting}
\end{Shaded}

Of course \texttt{Haskell} also has an if-then construct:

\begin{Shaded}
\begin{Highlighting}[]
    \NormalTok{max x y }\FunctionTok{=} 
        \KeywordTok{if} \NormalTok{x }\FunctionTok{>} \NormalTok{y}
            \KeywordTok{then} \NormalTok{x}
            \KeywordTok{else} \NormalTok{y}
\end{Highlighting}
\end{Shaded}

Again the \texttt{if ... then ... else ...} construct is an
\emph{expression}, so it returns a value.

\subsection{Case statement}\label{case-statement}

Many languages provide a \texttt{case} statement for conditions with
more than two choices. For example, Ruby provides a \texttt{case}
expression:

\begin{Shaded}
\begin{Highlighting}[]
    \DataTypeTok{Red} \NormalTok{= }\DecValTok{1}
    \DataTypeTok{Blue} \NormalTok{= }\DecValTok{2}
    \DataTypeTok{Yellow} \NormalTok{= }\DecValTok{3}

    \NormalTok{color = set_color();}
    \NormalTok{action = }\KeywordTok{case} \NormalTok{color }
        \KeywordTok{when} \DataTypeTok{Red} \KeywordTok{then} \NormalTok{action1()}
        \KeywordTok{when} \DataTypeTok{Blue} \KeywordTok{then} \NormalTok{action2()}
        \KeywordTok{when} \DataTypeTok{Yellow} \KeywordTok{then} \NormalTok{action3()}
    \KeywordTok{end}
\end{Highlighting}
\end{Shaded}

In \texttt{Haskell}, the case works and looks similar:

\begin{Shaded}
\begin{Highlighting}[]
    \KeywordTok{data} \DataTypeTok{Color} \FunctionTok{=} \DataTypeTok{Red} \FunctionTok{|} \DataTypeTok{Blue} \FunctionTok{|} \DataTypeTok{Yellow}

    \NormalTok{color }\FunctionTok{=} \NormalTok{set_color}
    \NormalTok{action }\FunctionTok{=} \KeywordTok{case} \NormalTok{color }\KeywordTok{of}
        \DataTypeTok{Red} \OtherTok{->} \NormalTok{action1}
        \DataTypeTok{Blue} \OtherTok{->} \NormalTok{action2}
        \DataTypeTok{Yellow} \OtherTok{->} \NormalTok{action3}
\end{Highlighting}
\end{Shaded}

Note however how we use the type as the value to decide on the case,
where in other languages we need to define some kind of enumeration.

\subsection{Generics/Templates}\label{genericstemplates}

In \texttt{Java} and \texttt{C++} there are generic data types (aka
template types), such as:

\begin{Shaded}
\begin{Highlighting}[]
    \NormalTok{Map<String,Integer> set = }\KeywordTok{new} \NormalTok{HashMap<String,Integer> ;}
\end{Highlighting}
\end{Shaded}

In \texttt{Haskell}, you would write this as follows:

\begin{Shaded}
\begin{Highlighting}[]
\OtherTok{    set ::} \DataTypeTok{Data.Map.Map} \DataTypeTok{String} \DataTypeTok{Integer} 
    \NormalTok{set }\FunctionTok{=} \NormalTok{Data.Map.empty}
\end{Highlighting}
\end{Shaded}

The main difference is of course that \texttt{set} in \texttt{Haskell}
is not an object but an immutable variable, so where in Java you would
say:

\begin{Shaded}
\begin{Highlighting}[]
    \NormalTok{set.}\FunctionTok{put}\NormalTok{(}\StringTok{"Answer"}\NormalTok{,}\DecValTok{42}\NormalTok{)}
\end{Highlighting}
\end{Shaded}

In \texttt{Haskell} you would say:

\begin{Shaded}
\begin{Highlighting}[]
    \NormalTok{set' }\FunctionTok{=} \NormalTok{insert }\StringTok{"Answer"} \DecValTok{42} \NormalTok{set}
\end{Highlighting}
\end{Shaded}

Because in \texttt{Haskell} variables are immutable, the return value of
the \texttt{insert} call is bound to a new variable rather than updating
the variable in place as in \texttt{Java}.

\end{document}